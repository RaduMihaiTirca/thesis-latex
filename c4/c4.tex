\chapter{Numerical Results}

\section{Filon Method}

The practical implementation of this method is done by using two functions. In accordance with \eqref{oscillatoryIntegral}, $I_1$ and $I_2$ are defined:


\begin{equation}
      I_1=\int_a^{b}dx cos(x)e^{i\omega x}
\end{equation}

\begin{equation}
      I_2 =\int_a^{b}dx (3x+x^2+x^3)e^{i\omega x} 
\end{equation}

These two integrals both have analytical solutions:

\begin{equation}
  I_1=\frac{e^{i a \omega } (\sin (a)+i \omega  \cos (a))-i e^{i b \omega } (\omega  \cos (b)-i \sin (b))}{\omega ^2-1}
\end{equation}

\begin{equation}
  \begin{aligned}
    I_2=\frac{1}{\omega^5}i (&e^{i a \omega } (a (a^3+a+3) \omega ^4+i (4 a^3+2 a+3) \omega ^3-2 (6 a^2+1) \omega ^2-24 i a \omega +24)\\
                             &-e^{i b \omega } (b (b^3+b+3) \omega ^4+i (4 b^3+2 b+3) \omega ^3-2 (6 b^2+1) \omega ^2-24 i b \omega +24))
  \end{aligned}
\end{equation}

The following numerical results were produced using "filonQuadrature.cpp" code,found in annex 1.



\begin{table}[h!]
    \begin{center}
      \caption{Relative error of trapezoidal method "$\delta_{Q^T}$" and Filon quadrature "$\delta_{Q^F}$" vs no. of steps $N$ on $I_1$ with $a=0,b=100,\omega=10$.}
      \label{table1}
      \pgfplotstabletypeset[
        multicolumn names, % allows to have multicolumn names
        col sep=comma, % the seperator in our .csv file
        display columns/0/.style={
          column name=$N$, % name of first column
          column type={S},string type},  % use siunitx for formatting
        display columns/1/.style={
          column name=$\delta_{Q^T}$,
          column type={S},string type},
        display columns/2/.style={
          column name=$\delta_{Q^F}$,
          column type={S},string type},
        every head row/.style={
          before row={\toprule}, % have a rule at top
          after row={
               & \% & \% \\ % the units seperated by &
              \midrule} % rule under units
              },
          every last row/.style={after row=\bottomrule}, % rule at bottom
      ]{c4/relErrorCosOmega10.csv} % filename/path to file
    \end{center}
  \end{table}


  \begin{table}[h!]
    \begin{center}
      \caption{Relative error of trapezoidal method "$\delta_{Q^T}$" and Filon quadrature "$\delta_{Q^F}$" vs no. of steps $N$ on $I_1$ with $a=0,b=100,\omega=10^3$.}
      \label{table2}
      \pgfplotstabletypeset[
        multicolumn names, % allows to have multicolumn names
        col sep=comma, % the seperator in our .csv file
        display columns/0/.style={
          column name=$N$, % name of first column
          column type={S},string type},  % use siunitx for formatting
        display columns/1/.style={
          column name=$\delta_{Q^T}$,
          column type={S},string type},
        display columns/2/.style={
          column name=$\delta_{Q^F}$,
          column type={S},string type},
        every head row/.style={
          before row={\toprule}, % have a rule at top
          after row={
               & \% & \% \\ % the units seperated by &
              \midrule} % rule under units
              },
          every last row/.style={after row=\bottomrule}, % rule at bottom
      ]{c4/relErrorCosOmega1000.csv} % filename/path to file
    \end{center}
  \end{table}

  \begin{table}[h!]
    \begin{center}
      \caption{Relative error of trapezoidal method "$\delta_{Q^T}$" and Filon quadrature "$\delta_{Q^F}$" vs no. of steps $N$ on $I_2$ with $a=0,b=100,\omega=10$.}
      \label{table3}
      \pgfplotstabletypeset[
        multicolumn names, % allows to have multicolumn names
        col sep=comma, % the seperator in our .csv file
        display columns/0/.style={
          column name=$N$, % name of first column
          column type={S},string type},  % use siunitx for formatting
        display columns/1/.style={
          column name=$\delta_{Q^T}$,
          column type={S},string type},
        display columns/2/.style={
          column name=$\delta_{Q^F}$,
          column type={S},string type},
        every head row/.style={
          before row={\toprule}, % have a rule at top
          after row={
               & \% & \% \\ % the units seperated by &
              \midrule} % rule under units
              },
          every last row/.style={after row=\bottomrule}, % rule at bottom
      ]{c4/relErrorPolyOmega10.csv} % filename/path to file
    \end{center}
  \end{table}

  \begin{table}[h!]
    \begin{center}
      \caption{Relative error of trapezoidal method "$\delta_{Q^T}$" and Filon quadrature "$\delta_{Q^F}$" vs no. of steps $N$ on $I_2$ with $a=0,b=100,\omega=10^3$.}
      \label{table4}
      \pgfplotstabletypeset[
        multicolumn names, % allows to have multicolumn names
        col sep=comma, % the seperator in our .csv file
        display columns/0/.style={
          column name=$N$, % name of first column
          column type={S},string type},  % use siunitx for formatting
        display columns/1/.style={
          column name=$\delta_{Q^T}$,
          column type={S},string type},
        display columns/2/.style={
          column name=$\delta_{Q^F}$,
          column type={S},string type},
        every head row/.style={
          before row={\toprule}, % have a rule at top
          after row={
               & \% & \% \\ % the units seperated by &
              \midrule} % rule under units
              },
          every last row/.style={after row=\bottomrule}, % rule at bottom
      ]{c4/relErrorPolyOmega1000.csv} % filename/path to file
    \end{center}
  \end{table}

\section{Generalized Filon Method}


%%% Local Variables:
%%% mode: latex
%%% TeX-master: "../th"
%%% End:
