% !TeX root = ../diss.tex
\chapter{Metoda Filon}
\label{cap:cap2}

\section{Numerical integration}
%Rectangle rule

\paragraph{} Numerical integration is comprised of a broad family of algorithms that aim to numerically compute values of finite integrals.
All can be traced back to this simplest form:

\begin{equation}
    \int_a^b f(x) dx \approx (b-a) f\left(\frac{a+b}{2}\right),\label{rectangle}
\end{equation}

which is known as "rectangle rule". This is the procedure trough which the area underneath a function, up to the $"x"$ axis, is approximated with a rectangle. This gives a very rough approximate of the actual integral


%Trapezoidal rule

A better approximation is given by this formula: 

\begin{equation}
    \int_a^b f(x) dx \approx (b-a) \left(\frac{f(a)+f(b)}{2}\right),\label{trapezoid}
\end{equation}

which approximates the area with a trapezoid, thus giving it the name "trapezoidal rule".

%Composite Trapezoidal rule

For both of these methods, the accuracy of the result can be virtually infinitely improved by dividing the $[a,b]$ interval in finer and finer $n$ subintervals.
This, in the case of the "trapezoidal rule" yields the following formula:

\begin{equation}
    \int_a^b f(x) dx \approx \frac{b-a}{n}\left(\frac{f(a)}{2}+\sum_{k=1}^{n-1}\left(f\left(a+k\frac{b-a}{n}\right)\right)+\frac{f(b)}{2}\right),\label{composite}
\end{equation}
where the sub intervals are of the form $ [a+kh,a+(k+1)h] \subset [a,b], h=\frac{b-a}{n} $.

These methods are generally referred to as "quadratures". The therm, "quadrature", has been historically used to describe the process of determining area. Now, it is somewhat loosely used to describe pretty much all numerical integration methods.

\section{Quadrature}

\paragraph{} Quadrature rules can be derived by constructing interpolating functions that are easy to integrate numerically.
The interpolating function can be a polynomial of degree zero (a constant function) like in the case of the rectangle rule\eqref{rectangle},
where the interpolation point is $\left(\frac{a+b}{2},f\left(\frac{a+b}{2}\right)\right)$. Similarly, for the trapezoidal rule\eqref{trapezoid}, the interpolating function is a polynomial of degree 1 (a straight line) that passes through two points $(a,f(a))$ and $(b,f(b))$. 
In the case of the composite rule\eqref{composite}, the number of interpolation points is directly proportional with the number of subintervals or steps, $n$. Interpolation with polynomials evaluated at equally spaced points yields a Newton-Cotes formula of which the previously mentioned methods ar a type.

\paragraph{} We have the general quadrature formula: 


\begin{equation}
    \int_a^b dx f(x)  \approx \sum_{i=1}^n w_if(x_i), \quad a \leq x_i \leq b,\label{quadrature}
\end{equation}
    
%\paragraph{Gausian quadrature}is the most common type of numerical quadrature. It is suitable to yield exact values for polynomials of degree $2n-1$,
%but can also offer 

%To be completed

\section{Oscillatory integrals with constant oscillation frequency}


The types of integrals that we address are of the form:

\begin{equation}
    I[a,b] = \int_a^b dx \varphi(x)e^{i\omega x}\label{oscillatoryIntegral}
\end{equation}

\paragraph{}Here we find $\varphi(x)$, the preexponential and our oscillatory component $e^{i\omega x}$ with the constant oscillation frequency $\omega$.
Under the assumption that $\varphi(x)$ is slowly oscillating, Filon suggests dividing the integration interval in $N$ subintervals and apply a 3-point quadrature scheme on each of these, where only the preexponential function $\varphi(x)$ is interpolated as opposed to the whole integrand.
Afterwards, the entire integral is approximated at $2n+1$ nodes. 

\paragraph{}Each subinterval is interpolated by a 3-point scheme:

\begin{equation}
    I(x_n,x_{n+2}) = \int_{x_n}^{x_{n+2}} dx \varphi(x)e^{i\omega x} \approx w_n\varphi_n+w_{n+1}\varphi_{n+1}+w_{n+2}\varphi_{n+2}\label{subintervalInterpolation}
\end{equation}

Where $\varphi(x)$ is approximated by a second degree polynomial, 

\begin{equation}
    \varphi(x) \approx c_0 + c_1x + c_2x^2\label{IntegrandAproximation}
\end{equation}

It is important to note that if this approximation is exact, the equation\eqref{subintervalInterpolation} gives an exact value for $I(x_n,x_{n+2})$.

The next step is to find the weights $w_n$. This is achieved by integrating over the three moments: 

\begin{equation}
    J_{l}=:\int_{x_{n}}^{x_{n+2}} x^{l} \mathrm{e}^{\mathrm{i} \omega x}=w_{n} x_{n}^{l}+w_{n+1} x_{n+1}^{l}+w_{n+2} x_{n+2}^{l} \quad l \in\{0,1,2\}
\end{equation}

It is an important assumption of the Filon method that the $J_l$ moments can be solved analyticall.  
After solving the system, we find: 

\begin{equation}
    \begin{aligned}
        w_{n+l} &=\zeta_{l} h \mathrm{e}^{\mathrm{i} \omega x_{n+l}} \\
        \zeta_{0} &=\zeta_{a}+\mathrm{i} \zeta_{b} \\
        \zeta_{a} &=\frac{3 \Theta+\cos (2 \Theta) \Theta-2 \sin (2 \Theta)}{2 \Theta^{3}} \\
        \zeta_{b} &=\frac{2 \Theta^{2}+2 \cos (2 \Theta)+\sin (2 \Theta) \Theta-2}{2 \Theta^{3}} \\
        \zeta_{1} &=\frac{4(\sin (\Theta)-\Theta \cos (\Theta))}{\Theta^{3}} \\
        \zeta_{2} &=\zeta_{a}-\mathrm{i} \zeta_{b}
    \end{aligned} \label{moments}
\end{equation}

with the parameter $\Theta$ defined as $\Theta=\omega h$ and $h$ being the step size ($h=\frac{b-a}{N}$).

\paragraph{}Inserting all this in the equation\eqref{subintervalInterpolation} and some further processing we find:

\begin{equation}
    \begin{aligned}
        \int_{x_{n}}^{x_{n+2}} d x \varphi(x) \mathrm{e}^{\mathrm{i} \omega x} \approx Q^{\text{F}}(x_n,x_{n+2})=
        & h\left[\mathrm{i} \alpha\left(\varphi_{n} \mathrm{e}^{\mathrm{i} \omega x_{n}}-\varphi_{n+2} \mathrm{e}^{\mathrm{i} \omega x_{n+2}}\right)\right.\\
        &+\beta \left(\varphi_{n} \mathrm{e}^{\mathrm{i} \omega x_{n}}+\varphi_{n+2} \mathrm{e}^{\mathrm{i} \omega x_{n+2}}\right) \\
        &\left.+\gamma \varphi_{n+1} \mathrm{e}^{\mathrm{i} \omega x_{n+1}}\right]
    \end{aligned}
\end{equation}
with $\alpha, \beta, \gamma$ defined as:
\begin{equation}
    \begin{aligned}
        &\alpha=\alpha(\Theta)=\frac{2 \Theta^{2}-2 \sin ^{2}(\Theta)+\sin (2 \Theta) \Theta}{2 \Theta^{3}}\\
        &\beta=\beta(\Theta)=\frac{2 \Theta\left(1+\cos ^{2}(\Theta)\right)-2 \sin (2 \Theta)}{2 \Theta^{3}}\\
        &\gamma=\gamma(\Theta)=\frac{4(-\Theta \cos (\Theta)+\sin (\Theta))}{\Theta^{3}}
    \end{aligned}
\end{equation}

It should be noted that for small $\Theta$, these parameters have the Taylor expansions:

\begin{equation}
    \begin{array}{l}
        \alpha(\theta)=\frac{2}{45} \theta^{3}-\frac{2}{313} \theta^{5}+\frac{2}{4725} \theta^{7}+\cdots \\
        \beta(\theta)=\frac{2}{3}+\frac{2}{15} \theta^{2}-\frac{4}{105} \theta^{4}+\frac{2}{567} \theta^{6}+\cdots \\
        \gamma(\theta)=\frac{4}{3}-\frac{2}{15} \theta^{2}+\frac{1}{210} \theta^{4}-\frac{1}{11340} \theta^{6}+\cdots
        \end{array}.
\end{equation}

The summation of these contributions over all the subintervals, brings us to the textbook formula for the Filon quadrature:

\begin{equation}
    \begin{aligned}
        \int_{a}^{b} d x \varphi(x) \mathrm{e}^{\mathrm{i} \omega x}=& \sum_{n=0}^{N-1} \int_{x_{2 n}}^{x_{2 n+2}} d x \varphi(x) \mathrm{e}^{\mathrm{i} \omega x} \approx \sum_{n=0}^{N-1} Q^{\text {F}}\left(x_{2 n}, x_{2 n+2}\right) \\
        =& h\left[\mathrm{i} \alpha\left(\varphi_{0} \mathrm{e}^{\mathrm{i} \omega x_{0}}-\varphi_{2 N} \mathrm{e}^{\mathrm{i} \omega x_{2 N}}\right)\right.\\
        &+\beta\left(\sum_{n=0}^{N} \varphi_{2 n} \mathrm{e}^{\mathrm{i} \omega x_{2 n}}-\frac{1}{2}\left[\varphi_{2 N} \mathrm{e}^{\mathrm{i} \omega x_{2 N}}+\varphi_{0} \mathrm{e}^{\mathrm{i} \omega x_{0}}\right]\right) \\
        &\left.+\gamma \sum_{n=1}^{N} \varphi_{2 n-1} \mathrm{e}^{\mathrm{i} \omega x_{2 n-1}}\right] \\
        &+\mathcal{O}\left(h^{4}\right)
    \end{aligned}
\end{equation}

%%% Local Variables:
%%% mode: latex
%%% TeX-master: "../th"
%%% End:
