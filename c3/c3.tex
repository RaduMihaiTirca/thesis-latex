\chapter{Filon Method Generalization}

\section{The issue}

\paragraph{} The Filon method in its pure form is only suited for \eqref{oscillatoryIntegral}.
In real-world applications it isn't very often that these type of functions are found. The problem arises with the fact that the $\theta$ \eqref{moments} parameter depends on the oscillation frequency $\omega$ which needs to be constant.
This means that the oscillation frequency is going to be a function of its on. This function wil have to be monotonous such that the solution that is to be provided is going to work. 

\section{Univariate Dynamic Integral}

The now expanded\cite{art2} form of the integrals that are to be calculated is:

\begin{equation}
    \int_{a}^{b}dxA(x)e^{ih(x)}
\end{equation}

where $h(x)$ is the new, variable oscillation frequency.

In order to be able to compute this type of integral, we need to bring it into a Filon compatible form. This means that we have to reparameterize $h(x)$ onto a variable that rises linearly from its minimal value $h(x_{min})$, up to its maximal value $h(x_{max})$.

\begin{equation}
    h(x)=y \label{parameterization}   
\end{equation}

This is achieved by applying a change of variable:

\begin{equation}
    y=h(x), \quad \dd y=\dv{f}{x} \dd x, \quad \dd x=\frac{\dd y}{\left(\dv{f}{x}\right)} \label{changeOfVariable}
\end{equation}

Having previously imposed the constraint of monotony, it is now clear that the $\dv{f}{x}$ denominator will not vanish.

\begin{equation}
    \int_{y_a}^{y_b}dy\frac{A(x(y))}{\dv{f}{x}\left(x(y)\right)}e^{iy}
\end{equation}

In this new form of the integral, another issue arises from a numerical stand point. 
The Filon method requires an equidistant array of values as integration domain.

\begin{equation}
    x = [x_1, x_2, x_3, ..., x_N]; \quad x_1 = a, x_N = b; \quad step = \Delta x = \frac{b-a}{n}. 
\end{equation}\\

The newly obtained array: 
\begin{equation}
    y_t=h(x)=[h(x_1),h(x_2),h(x_3), ..., h(x_N)]; \quad y_a=h(x_1), y_b=h(x_N) 
\end{equation}

is evidently not an equidistant one, but inherits the monotony of its generator function $h(x)$, such that:
$$
    y_{t1}=h(x_1) < y_{t2}=h(x_2) < y_{t3}=h(x_3) < ... < y_{tN}=h(x_N).
$$

The next step is to find the $x(y)$ array, bound by $y_a$ and $y_b$ such that:

\begin{equation}
        y = [y_1, y_2, y_3, ..., y_N]; \quad y_N-y_{N-1}=\Delta y = \frac{y_b-y_a}{N} 
\end{equation}

\begin{equation}
    x(y)=h^{-1}(y)=[h^{-1}(y_1), h^{-1}(y_2, h^{-1}(y_3), ...,h^{-1}(y_N)] 
\end{equation}

\begin{equation}
    h(x(y)) = [h_1,h_2,h_3, ..., h_N];\quad h_N-h_{N-1}=\Delta h = \frac{h_N-h_1}{N} 
\end{equation}

This can be done using linear interpolation of $y$ between $y_t$ and $x$.

\begin{equation}
    x(y)=x_n+(y-y_{tn})\frac{x_{n+1}-x_n}{y_{tn+1}-y_{tn}}.
\end{equation}

Inf we define:
\begin{equation}
    \Phi_i \equiv \frac{A(x(y_i))}{\dv{f}{x}\left(x(y_i)\right)},
\end{equation}

the \eqref{FilonQ} formula becomes:

\begin{equation}
    \begin{aligned}
        \int_{a}^{b}dxA(x)e^{ih(x)} &= \int_{y_a}^{y_b}dy\frac{A(x(y))}{\dv{f}{x}\left(x(y)\right)}e^{iy} \approx \sum_{n=0}^{N-1} Q^{\text {F}}\left(y_{2 n}, y_{2 n+2}\right) \\
        &= h\left[\mathrm{i} \alpha\left(\Phi_{0} \mathrm{e}^{\mathrm{i}  y_{0}}-\Phi_{2 N} \mathrm{e}^{\mathrm{i}  y_{2 N}}\right)\right.\\
        &+\beta\left(\sum_{n=0}^{N} \Phi_{2 n} \mathrm{e}^{\mathrm{i}  y_{2 n}}-\frac{1}{2}\left[\Phi_{2 N} \mathrm{e}^{\mathrm{i}  y_{2 N}}+\Phi_{0} \mathrm{e}^{\mathrm{i}  y_{0}}\right]\right) \\
        &\left.+\gamma \sum_{n=1}^{N} \Phi_{2 n-1} \mathrm{e}^{\mathrm{i}  y_{2 n-1}}\right] \\
    \end{aligned} \label{FilonVarQ}
\end{equation} 


where $\alpha, \beta, \gamma$ are still defined by \eqref{abg}, but  $\theta=h=\frac{y_b-y_a}{N}$. 

%%% Local Variables:
%%% mode: latex
%%% TeX-master: "../th"
%%% End:
