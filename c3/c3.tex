\chapter{Filon Method Generalization}

\section{The issue}

\paragraph{} The Filon method in its pure form is only suited for \eqref{oscillatoryIntegral}.
In real-world applications it isn't very often that these type of functions are found. The problem arises with the fact that the $\theta$ \eqref{moments} parameter depends on the oscillation frequency $\omega$ which needs to be constant.
This means that the oscillation frequency is going to be a function of its on. This function wil have to be monotonous such that the solution that is to be provided is going to work. 

\section{Univariate Dynamic Integral}

The now expanded form of the integrals that are to be calculated is:

\begin{equation}
    \int_{a}^{b}dxA(x)e^{ih(x)}
\end{equation}

where $h(x)$ is the new, variable oscillation frequency.

In order to be able to compute this type of integral, we need to bring it into a Filon compatible form. This means that we have to reparameterize $h(x)$ onto a variable that rises linearly from its minimal value $h(x_{min})$, up to its maximal value $h(x_{max})$.

\begin{equation}
    h(x)=y \label{parameterization}   
\end{equation}

This is achieved by applying a change of variable:

\begin{equation}
    y=h(x), \quad \dd y=\dv{f}{x} \dd x, \quad \dd x=\frac{\dd y}{\left(\dv{f}{x}\right)} \label{changeOfVariable}
\end{equation}

Having previously imposed the constraint of monotony, it is now clear that the $\dv{f}{x}$ denominator won't vanish.

\begin{equation}
    \int_{y_a}^{y_b}dy\frac{A(x(y))}{\dv{f}{x}\left(x(y)\right)}e^{iy}
\end{equation}



%%% Local Variables:
%%% mode: latex
%%% TeX-master: "../th"
%%% End:
